%!TEX program = xelatex
\documentclass[12pt, a4paper]{article}

\usepackage{fancyhdr}
\usepackage{extramarks}
\usepackage{amssymb}
\usepackage{amsmath}
\usepackage[hidelinks]{hyperref}
\usepackage{gensymb} % For the °(degree) symbol, \degree
\usepackage{bm}
\usepackage{centernot}
\usepackage{mathtools}
\usepackage{graphicx}
\usepackage{grffile}
\usepackage{float}
\usepackage{breqn}
\usepackage{algorithm}
\usepackage{algpseudocode}
\usepackage{dsfont}
\usepackage[table]{xcolor}
\usepackage{tikz}
\usetikzlibrary{shapes,arrows,arrows.meta,decorations.pathmorphing}
\usetikzlibrary{positioning}
\usetikzlibrary{shadows}
\usepackage{adjustbox}
\usepackage{enumerate}
\usepackage{extarrows}
\usepackage{soul}
\usepackage{array}
\usepackage{calc}
\usepackage{hhline,colortbl}
\pagestyle{empty}% for cropping
\usepackage{tikz}
\usetikzlibrary{calc}
\usepackage{zref-savepos}
\usepackage{tabu}
\usepackage[most]{tcolorbox}
\usepackage{scrextend}
\tcbuselibrary{breakable}
\usepackage{ifthen}
\usepackage{tabularx}
\usepackage{enumitem}
\usepackage{booktabs}
\usepackage{caption}
\usepackage[labelformat=parens]{subcaption}
\usepackage{geometry}

\usepackage{draftfigure}

\usepackage{ifxetex}
\ifxetex
\usepackage{fontspec}
\defaultfontfeatures{Ligatures=TeX}
%\setmonofont{Monaco}
%\else
%\usepackage{tgtermes}
\fi

%% Page layout
%\geometry{a4paper,margin=1.0in}

\setmainfont{SourceSerifPro}[
Path=../../fonts/source_serif_pro/,
UprightFont={*-Regular},
ItalicFont={*-Italic},
BoldFont={*-Bold},
BoldItalicFont={*-BoldItalic},
Extension=.ttf
]

%\setmainfont{lora}[
%Path=../../fonts/lora/,
%UprightFont={*_regular},
%BoldFont={*_bold},
%ItalicFont={*_italic},
%BoldItalicFont={*_bold_italic},
%Extension=.ttf
%]

\newfontfamily{\lora}[
Path=../../fonts/lora/,
UprightFont={*_regular},
BoldFont={*_bold},
ItalicFont={*_italic},
BoldItalicFont={*_bold_italic},
Extension=.ttf
]
{lora}

\newfontfamily{\notosans}
[Ligatures=TeX, % recommended
Path=../../fonts/notosans/,
Extension=.ttf,
UprightFont={*_regular},
ItalicFont={*_italic},
BoldFont={*_bold},
BoldItalicFont={*_bold_italic}]
{notosans}

\newfontfamily{\notosanssc}
[Ligatures=TeX,
Path=../../fonts/noto_sans_sc/,
Extension=.otf,
UprightFont={*_regular},
BoldFont={*_bold}]
{noto_sans_sc}

\newfontfamily{\notoserif}[
Path=../../fonts/noto_serif/,
UprightFont={*-Regular},
ItalicFont={*-Italic},
BoldFont={*-Bold},
BoldItalicFont={*-BoldItalic},
Extension=.ttf
]{NotoSerif}

%\setmainfont{NotoSerif}
%[
%Path=../../fonts/noto_serif/,
%UprightFont={*-Regular},
%ItalicFont={*-Italic},
%BoldFont={*-Bold},
%BoldItalicFont={*-BoldItalic},
%Extension=.ttf
%]

%\setmainfont{TimesNewerRoman}
%[
%Path=../../fonts/times_newer_roman/,
%UprightFont={*-Regular},
%ItalicFont={*-Italic},
%BoldFont={*-Bold},
%BoldItalicFont={*-BoldItalic},
%Extension=.otf
%]

\newfontfamily{\timesnewerroman}[
Path=../../fonts/times_newer_roman/,
UprightFont={*-Regular},
ItalicFont={*-Italic},
BoldFont={*-Bold},
BoldItalicFont={*-BoldItalic},
Extension=.otf
]{TimesNewerRoman}

\newfontfamily{\philosopher}
[Ligatures=TeX,
Path=../../fonts/philosopher/,
Extension=.ttf,
UprightFont={*-Regular},
ItalicFont={*-Italic},
BoldFont={*-Bold},
BoldItalicFont={*-BoldItalic}]
{Philosopher}

\newfontfamily{\raleway}
[Ligatures=TeX,
Path=../../fonts/raleway/,
Extension=.ttf,
UprightFont={*-Regular},
ItalicFont={*-Italic},
BoldFont={*-Bold},
BoldItalicFont={*-BoldItalic}]
{Raleway}

\newfontfamily{\barlow}
[Ligatures=TeX, % recommended
Path=../../fonts/barlow/,
Extension=.ttf,
UprightFont={*_regular},
ItalicFont={*_italic},
BoldFont={*_bold},
BoldItalicFont={*_bold_italic}]
{barlow}

\definecolor{MyTitleColor}{HTML}{2c82c9}
\definecolor{MyLinkColor}{HTML}{2c82c9}

\hypersetup{
    colorlinks=true,
    citecolor=black,
    linkcolor=black,
    filecolor=black,
    urlcolor=MyLinkColor
}

%\hypersetup{
%	colorlinks=true,
%	urlcolor=MyLinkColor
%}

%\hypersetup{linkcolor=black, urlcolor=black}

\usepackage{xeCJK}
\setCJKmainfont{source_han_serif_cn}
[
Path=../../fonts/source_han_serif_cn/,
UprightFont={*_regular},
BoldFont={*_bold},
Extension=.otf
]

% Page layout
\usepackage{geometry}
%\geometry{a4paper,margin=1.0in}
%\geometry{
%	    a4paper,
%	    landscape,
%	    width=297mm,
%	    height=210mm,
%	    left=2mm,
%	    top=2mm,
%	    right=2mm,
%	    bottom=2mm
%	}
\geometry{
	a4paper,
	landscape,
	margin=1.0in
}

% Set the indention of paragraphs to zero
\setlength{\parindent}{0pt}

% Set the paragraph spacing
\setlength{\parskip}{1em}

\newcolumntype{C}[1]{>{\centering\let\newline\\\arraybackslash\hspace{0pt}}m{#1}}

\begin{document}

\begin{center}
    \vspace*{2cm}
    {\textcolor{MyTitleColor}{
            \fontsize{40}{48}\philosopher\selectfont
            G-LEAP Journal Experiments Report}}\\[\baselineskip]
    \vspace*{1cm}
    {\fontsize{20}{24}\barlow\selectfont 2022.1}\\[\baselineskip]
    \vspace*{3cm}
    {\fontsize{16}{19.2}\selectfont Jianfeng Hou}\\[\baselineskip]
    {\fontsize{16}{19.2}\selectfont\hypersetup{linkcolor=black, urlcolor=black} \href{mailto:houjf@shanghaitech.edu.cn}{houjf@shanghaitech.edu.cn}}\\[\baselineskip]
    \vfill
\end{center}

\thispagestyle{empty}

\newpage

\pagenumbering{arabic}

\section{Emoji Prediction Datasets}

We train the emoji prediction models using different datasets (at least not one single dataset) for the following two reasons:
\begin{enumerate}
	\item In real-world scenarios, the models have already been pre-trained when deploying our edge-assisted emoji prediction system. The emoji prediction models may be trained by different groups of people using different devices (cloud servers, edge servers, or even end-devices). The collected dataset for training one emoji prediction model probably differ from others.
	
	\item Using different training datasets can increase the diversity of our emoji prediction models.
\end{enumerate}

We use the following emoji prediction datasets:
\begin{enumerate}
	\item Celebrity Profiling (Emoji Prediction: Extensions and Benchmarking)
	
	\item Twitter (SemEval-2018 Task 2)
	
	\item Twitter (\href{https://www.kaggle.com/hariharasudhanas/twitter-emoji-prediction}{Twitter Emoji Prediction | Kaggle})
	
	\item Twitter (DeepMoji), \textbf{\textcolor{red}{TODO}}
\end{enumerate}

\section{Emoji Prediction Models}

The emoji prediction models can differ in the following three dimensions:
\begin{enumerate}
	\item Base Model;
	\item Vectorization Method;
	\item Training Dataset.
\end{enumerate}

Thus, the trained emoji prediction models used in our experiments can be listed clearly in Table~\ref{table:emoji_prediction_models}.

\begin{table}[h]
	\centering
	\caption{The trained emoji prediction models used in our experiments.}
	\label{table:emoji_prediction_models}
	\begin{tabular}{c|c|c|c|c}
		\hline
		\textbf{Index} & \textbf{Base Model} & \textbf{Validation Accuracy} & \textbf{Model Size} & \textbf{File} \\
		\hline
		0 & SVM & 10462/50000 = 20.92\% & 768K & \texttt{statistical/svm.pkl} \\
		1 & Na\"ive Bayes & 11110/50000 = 22.22\% & 1.3M & \texttt{statistical/naive\_bayes.pkl} \\
		2 & Decision Tree & 6568/50000 = 13.14\% & 1.1M & \texttt{statistical/decision\_tree.pkl} \\
		\hline
		3 & RNN-1 & 12781/50000 = 25.56\% &  & \texttt{neural/rnn/rnn\_1.pkl} \\
		4 & RNN-2 & 13265/50000 = 26.53\% &  & \texttt{neural/rnn/rnn\_2.pkl} \\
		\hline
		5 & LSTM-1 & 14255/50000 = 28.51\% &  & \texttt{neural/lstm/lstm\_1.pkl} \\
		6 & LSTM-2 & 14677/50000 = 29.35\% &  & \texttt{neural/lstm/lstm\_2.pkl} \\
		7 & Bi-LSTM & 15342/50000 = 30.68\% &  & \texttt{neural/lstm/bi\_lstm.pkl} \\
		\hline
		8 & BERT & 16532/50000 = 33.06\% & 518M & \texttt{neural/transformer/bert.pkl} \\
		9 & RoBERTa & 16441/50000 = 32.88\% & 477M & \texttt{neural/transformer/roberta.pkl} \\
		\hline
	\end{tabular}
\end{table}

\begin{table}[h]
	\centering
	\caption{The trained emoji prediction models used in our experiments.}
	\label{table:selected_emoji_prediction_models}
	\begin{tabular}{c|c|c|c|c|c|c}
		\hline
		\textbf{Index} & \textbf{Base Model} & \textbf{Vectorization Method} & \textbf{Training Dataset} & \textbf{Model Size} & \textbf{Test Accuracy} & \textbf{Inference Time} \\
		\hline
		0 & Na\"ive Bayes &  &  &  &  \\
		1 & SVM &  &  &  &  \\
		2 & BERT &  &  &  &  \\
		3 & RoBERTa &  &  &  &  \\
		4 & MLP &  &  &  &  \\
		5 & LSTM &  &  &  &  \\
		6 & Bi-LSTM &  &  &  &  \\
		7 & RNN &  &  &  &  \\
		8 & DeepMoji &  &  &  &  \\
		9 & fastText &  &  &  &  \\
		\hline
	\end{tabular}
\end{table}

\subsection{Statistical Models}

Currently we only use Bag of Words (BoW) as the vectorization method for training all the statistical models.

\subsubsection{Na\"ive Bayes}

\subsubsection{Support Vector Machine (SVM)}

\subsubsection{Decision Tree}

\subsection{Neural Models}

In our experiments, we use the following vectorization method for training MLP models:
\begin{itemize}
	\item Word embeddings: GloVe embeddings.
	\item Sequence embeddings: mean of all words in a sentence. \href{https://edumunozsala.github.io/BlogEms/jupyter/nlp/classification/embeddings/python/2020/08/15/Intro_NLP_WordEmbeddings_Classification.html}{Reference}
\end{itemize}

\subsubsection{Multi-Layer Perceptron (MLP)}

\subsubsection{Recurrent Neural Network (RNN)}

\subsubsection{Long-Short Term Memory (LSTM)}

\subsubsection{Transformers}

\subsection{Vectorization}

\begin{enumerate}
	\item Bag of Words / Characters
	\item One-hot encoding
	\item word embeddings + sequence embeddings
\end{enumerate}

\section{Miscellaneous}

\subsection{Experiments Observations}

Neural models with smaller batch size can converge quicker, and produce higher accuracy.

\end{document}
