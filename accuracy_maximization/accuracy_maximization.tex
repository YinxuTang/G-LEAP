%!TEX program = xelatex
\documentclass[12pt, a4paper]{article}

\usepackage{fancyhdr}
\usepackage{extramarks}
\usepackage{amssymb}
\usepackage{amsmath}
\usepackage[hidelinks]{hyperref}
\usepackage{gensymb} % For the °(degree) symbol, \degree
\usepackage{bm}
\usepackage{centernot}
\usepackage{mathtools}
\usepackage{graphicx}
\usepackage{grffile}
\usepackage{float}
\usepackage{breqn}
\usepackage{algorithm}
\usepackage{algpseudocode}
\usepackage{dsfont}
\usepackage[table]{xcolor}
\usepackage{tikz}
\usetikzlibrary{shapes,arrows,arrows.meta,decorations.pathmorphing}
\usetikzlibrary{positioning}
\usetikzlibrary{shadows}
\usepackage{adjustbox}
\usepackage{enumerate}
\usepackage{extarrows}
\usepackage{soul}
\usepackage{array}
\usepackage{calc}
\usepackage{hhline,colortbl}
\pagestyle{empty}% for cropping
\usepackage{tikz}
\usetikzlibrary{calc}
\usepackage{zref-savepos}
\usepackage{tabu}
\usepackage[most]{tcolorbox}
\usepackage{scrextend}
\tcbuselibrary{breakable}
\usepackage{ifthen}
\usepackage{tabularx}
\usepackage{enumitem}
\usepackage{booktabs}
\usepackage{caption}
\usepackage[labelformat=parens]{subcaption}
\usepackage{geometry}

\usepackage{draftfigure}

\usepackage{ifxetex}
\ifxetex
\usepackage{fontspec}
\defaultfontfeatures{Ligatures=TeX}
\setmonofont{Monaco}
%\else
%\usepackage{tgtermes}
\fi

%% Page layout
%\geometry{a4paper,margin=1.0in}

\setmainfont{SourceSerifPro}[
Path=../fonts/source_serif_pro/,
UprightFont={*-Regular},
ItalicFont={*-Italic},
BoldFont={*-Bold},
BoldItalicFont={*-BoldItalic},
Extension=.ttf
]

%\setmainfont{lora}[
%Path=../../fonts/lora/,
%UprightFont={*_regular},
%BoldFont={*_bold},
%ItalicFont={*_italic},
%BoldItalicFont={*_bold_italic},
%Extension=.ttf
%]

\newfontfamily{\lora}[
Path=../fonts/lora/,
UprightFont={*_regular},
BoldFont={*_bold},
ItalicFont={*_italic},
BoldItalicFont={*_bold_italic},
Extension=.ttf
]
{lora}

\newfontfamily{\notosans}
[Ligatures=TeX, % recommended
Path=../fonts/notosans/,
Extension=.ttf,
UprightFont={*_regular},
ItalicFont={*_italic},
BoldFont={*_bold},
BoldItalicFont={*_bold_italic}]
{notosans}

\newfontfamily{\notosanssc}
[Ligatures=TeX,
Path=../fonts/noto_sans_sc/,
Extension=.otf,
UprightFont={*_regular},
BoldFont={*_bold}]
{noto_sans_sc}

\newfontfamily{\notoserif}[
Path=../fonts/noto_serif/,
UprightFont={*-Regular},
ItalicFont={*-Italic},
BoldFont={*-Bold},
BoldItalicFont={*-BoldItalic},
Extension=.ttf
]{NotoSerif}

%\setmainfont{NotoSerif}
%[
%Path=../../fonts/noto_serif/,
%UprightFont={*-Regular},
%ItalicFont={*-Italic},
%BoldFont={*-Bold},
%BoldItalicFont={*-BoldItalic},
%Extension=.ttf
%]

%\setmainfont{TimesNewerRoman}
%[
%Path=../../fonts/times_newer_roman/,
%UprightFont={*-Regular},
%ItalicFont={*-Italic},
%BoldFont={*-Bold},
%BoldItalicFont={*-BoldItalic},
%Extension=.otf
%]

\newfontfamily{\timesnewerroman}[
Path=../fonts/times_newer_roman/,
UprightFont={*-Regular},
ItalicFont={*-Italic},
BoldFont={*-Bold},
BoldItalicFont={*-BoldItalic},
Extension=.otf
]{TimesNewerRoman}

\newfontfamily{\philosopher}
[Ligatures=TeX,
Path=../fonts/philosopher/,
Extension=.ttf,
UprightFont={*-Regular},
ItalicFont={*-Italic},
BoldFont={*-Bold},
BoldItalicFont={*-BoldItalic}]
{Philosopher}

\newfontfamily{\raleway}
[Ligatures=TeX,
Path=../fonts/raleway/,
Extension=.ttf,
UprightFont={*-Regular},
ItalicFont={*-Italic},
BoldFont={*-Bold},
BoldItalicFont={*-BoldItalic}]
{Raleway}

\newfontfamily{\barlow}
[Ligatures=TeX, % recommended
Path=../fonts/barlow/,
Extension=.ttf,
UprightFont={*_regular},
ItalicFont={*_italic},
BoldFont={*_bold},
BoldItalicFont={*_bold_italic}]
{barlow}

\definecolor{MyTitleColor}{HTML}{2c82c9}
\definecolor{MyLinkColor}{HTML}{2c82c9}

\hypersetup{
    colorlinks=true,
    citecolor=black,
    linkcolor=black,
    filecolor=black,
    urlcolor=MyLinkColor
}

%\hypersetup{
%	colorlinks=true,
%	urlcolor=MyLinkColor
%}

%\hypersetup{linkcolor=black, urlcolor=black}

\usepackage{xeCJK}
\setCJKmainfont{source_han_serif_cn}
[
Path=../fonts/source_han_serif_cn/,
UprightFont={*_regular},
BoldFont={*_bold},
Extension=.otf
]

% Page layout
\usepackage{geometry}
\geometry{a4paper, margin=1.0in}
%\geometry{
%	    a4paper,
%	    landscape,
%	    width=297mm,
%	    height=210mm,
%	    left=2mm,
%	    top=2mm,
%	    right=2mm,
%	    bottom=2mm
%	}
%\geometry{
%	a4paper,
%	landscape,
%	margin=1.0in
%}

% Set the indention of paragraphs to zero
\setlength{\parindent}{0pt}

% Set the paragraph spacing
\setlength{\parskip}{1em}

\newcolumntype{C}[1]{>{\centering\let\newline\\\arraybackslash\hspace{0pt}}m{#1}}

\begin{document}

\begin{center}
    \vspace*{2cm}
    {\textcolor{MyTitleColor}{
            \fontsize{40}{48}\philosopher\selectfont
            Accuracy Maximization}}\\[\baselineskip]
    \vspace*{1cm}
    {\fontsize{20}{24}\barlow\selectfont 2022.1}\\[\baselineskip]
    \vspace*{3cm}
    {\fontsize{16}{19.2}\selectfont Jianfeng Hou}\\[\baselineskip]
    {\fontsize{16}{19.2}\selectfont\hypersetup{linkcolor=black, urlcolor=black} \href{mailto:houjf@shanghaitech.edu.cn}{houjf@shanghaitech.edu.cn}}\\[\baselineskip]
    \vfill
\end{center}

\thispagestyle{empty}

\newpage

\pagenumbering{arabic}

\section{Problem Formulation}

\subsection{Action (Super Arm)}

There are totally $N + 1$ arms.

In each round, the agent chooses an action (\textit{i.e.}, selects a super arm). Denote the set of all the super arms as $\mathcal{A}$.

\subsection{Policy}

A \textbf{policy} $\pi$ is a distribution over actions. Formally, a policy $\pi$ is defined as
\[
\pi (a) = \mathbb{P} \left[ A(t) = a \right], \forall a \in \mathcal{A}.
\]

Not that $ A(t)$ (\textit{i.e.}, the action in round $t$) can be equivalently represented by a decision vector $\bm{s} (t) = [ s_0 (t), s_1 (t), \dots, s_N (t) ]$, where
\[
s_i (t) = 
\begin{cases}
	1 &\text{ if } i \in A(t),\\
	0 &\text{ otherwise.}
\end{cases}
\]

\subsection{Long-Term Time-Averaged Constraint}

Under policy $\pi$, the energy consumption is \textit{i.i.d.} over rounds with the following mean:
\begin{align}
	\begin{split}
	\mathbb{E}_{\pi} \left[\sum_{i \in A(t)} E_i(t) \right] &= \sum\limits_{a \in \mathcal{A}} \mathbb{E} \left[ \sum\limits_{i \in A(t)} E_i (t) \bigg\vert A (t) = a \right] \cdot \pi(a)\\
	&= \sum\limits_{a \in \mathcal{A}} \mathbb{E} \left[ \sum\limits_{i \in a} E_i (t) \right] \cdot \pi(a)\\
	&= \sum\limits_{a \in \mathcal{A}} \sum\limits_{i \in a} \mathbb{E} \left[ E_i (t) \right] \cdot \pi(a)\\
	&= \sum\limits_{a \in \mathcal{A}} \rho_a \cdot \pi(a),
	\label{eq:energy_consumption_representation}
	\end{split}
\end{align}
where $\rho_a = \sum\limits_{i \in a} \mathbb{E} \left[ E_i (t) \right]$ for all $a \in \mathcal{A}$.

There exists a long-term time-averaged constraint on the energy consumption of the smartphone:
\begin{align}
	\mathop{\lim \sup}_{T' \to +\infty} \frac{1}{T'} \sum_{t=1}^{T'} \mathbb{E}_{\pi} \left[\sum_{i \in \mathcal{N}} E_i(t) s_i(t) \right] \leq b
	\label{eq:energy_consumption_constraint}
\end{align}
where $b$ is a positive constant representing the energy budget in each round.

According to \eqref{eq:energy_consumption_representation}, constraint~\eqref{eq:energy_consumption_constraint} can be written as:
\begin{align}
	\sum\limits_{a \in \mathcal{A}} \rho_a \cdot \pi(a) \leq b.
	\label{eq:energy_consumption_constraint_transformed}
\end{align}

\subsection{The Original Problem}

\[
K (t) = 1 - \prod\limits_{i \in \mathcal{N}} \left( 1 - C_i (t) s_i (t) \right) = 1 - \prod\limits_{i \in A(t)} \left( 1 - C_i (t) \right).
\]

%\textcolor{red}{Attention: No independence ($s_i (t)$ are not independent with each other).}
%\begin{align*}
%	\mathbb{E}_{\pi} \left[ K (t) \right] &= \mathbb{E}_{\pi} \left[ 1 - \prod\limits_{i \in \mathcal{N}} \left( 1 - C_i (t) s_i (t) \right) \right]\\
%	&= 1 - 
%\end{align*}

\begin{equation}
	\begin{aligned}
		\mathop {\rm{maximize}}_{\pi} \quad & \mathbb{E}_{\pi} \left[ \sum\limits_{t = 1}^{T} K (t) \right]\\
		\rm{subject~to} \quad & \sum\limits_{a \in \mathcal{A}} \rho_a \cdot \pi(a) \leq b.
	\end{aligned}
\end{equation}

For all $a \in \mathcal{A}$, the mean reward $\phi_a$ of pulling super arm $a$ is given by
\begin{align*}
	\phi_a &\triangleq \mathbb{E} \left[ R_a (t) \right]\\
	&= \mathbb{E} \Big[ K(t) \Big\vert A(t) = a \Big]\\
	&= \mathbb{P} \Big[ K(t) = 1 \Big\vert A(t) = a \Big]\\
	&= \mathbb{P} \left[ \prod\limits_{i \in A(t)} \left( 1 - C_i (t) \right) = 0 \bigg\vert A(t) = a \right]\\
	&= \mathbb{P} \left[ \prod\limits_{i \in a} \left( 1 - C_i (t) \right) = 0 \right]\\
	&= \mathbb{P} \left[ \bigcup\limits_{i \in a} C_i (t) = 1 \right]\\
	&= 1 - \mathbb{P} \left[ \bigcap\limits_{i \in a} C_i (t) = 0 \right]\\
	&= 1 - \prod\limits_{i \in a} (1 - c_i).
\end{align*}

The mean reward obtained by the agent under policy $\pi$ is given by
\[
\mathbb{E}_{\pi} \left[ R (t) \right] = \sum\limits_{a \in \mathcal{A}} \phi_a \cdot \pi (a).
\]

According to the INFOCOM 2019 Fairness paper (Jia Liu), assuming the mean reward vector $\bm{\phi} = \{ \phi_a: a \in \mathcal{A} \}$ is known in advance, the reward maximization problem with a long-term time-averaged constraint can be formulated as the following linear program:
\begin{align}
	\begin{split}
		\mathop {\rm{maximize}}_{\pi} \quad & \sum\limits_{a \in \mathcal{A}} \phi_a \cdot \pi (a)\\
		\rm{subject~to} \quad & \sum\limits_{a \in \mathcal{A}} \rho_a \cdot \pi(a) \leq b,\\
		& \pi(a) \in [0, 1], \forall a \in \mathcal{A},\\
		& \sum\limits_{a \in \mathcal{A}} \pi(a) = 1.
		\label{eq:optimization_problem}
	\end{split}
\end{align}

\subsection{The Transformed Problem}

\[
\begin{aligned}
	\mathop {\rm{maximize}}_{\pi} \quad & \mathbb{E}_{\pi} \left[ \sum\limits_{t = 1}^{T} \sum\limits_{i \in A(t)} C_i (t) \right]\\
	\rm{subject~to} \quad & \sum\limits_{a \in \mathcal{A}} \rho_a \cdot \pi(a) \leq b.
\end{aligned}
\]

For all $a \in \mathcal{A}$, the mean reward $\omega_a$ of pulling super arm $a$ is given by
\begin{align}
	\omega_a &\triangleq \mathbb{E} \left[ R_a^' (t) \right]\\
	&= \mathbb{E} \left[ \sum\limits_{i \in A(t)} C_i (t) \Big\vert A(t) = a \right]\\
	&= \sum\limits_{i \in a} c_i.
\end{align}

The mean reward obtained by the agent under policy $\pi$ is given by
\begin{align*}
	\mathbb{E}_{\pi} \left[ R' (t) \right] &= \sum\limits_{a \in \mathcal{A}} \mathbb{E} \left[ R^'_a (t) \right] \cdot \pi (a)\\
	&= \sum\limits_{a \in \mathcal{A}} \sum\limits_{i \in a} c_i \cdot \pi(a)\\
	&= \sum\limits_{a \in \mathcal{A}} \omega_a \cdot \pi(a).
\end{align*}

\begin{align}
	\begin{split}
		\mathop {\rm{maximize}}_{\pi} \quad & \sum\limits_{a \in \mathcal{A}} \omega_a \cdot \pi (a)\\
		\rm{subject~to} \quad & \sum\limits_{a \in \mathcal{A}} \rho_a \cdot \pi(a) \leq b,\\
		& \pi(a) \in [0, 1], \forall a \in \mathcal{A},\\
		& \sum\limits_{a \in \mathcal{A}} \pi(a) = 1.
		\label{eq:optimization_problem_transformed}
	\end{split}
\end{align}

\subsection{Proof of Equivalence}

Now we prove that the optimal solution to Problem~\ref{eq:optimization_problem_transformed} is exactly the optimal solution to Problem~\ref{eq:optimization_problem}.

\[
\mathbb{E}_{\pi_1} \left[ R^' (t) \right] \ge \mathbb{E}_{\pi_2} \left[ R^' (t) \right] \Rightarrow \mathbb{E}_{\pi_1} \left[ R (t) \right] \ge \mathbb{E}_{\pi_2} \left[ R (t) \right], \forall \pi_1, \pi_2 \in \mathcal{F}.
\]

\textbf{Proof}

As the constraint of Problem~\eqref{eq:optimization_problem} is exactly the same as the constraint of Problem~\eqref{eq:optimization_problem_transformed}, we can denote the set of all the feasible policies as $\mathcal{F}$. Assume $\pi^*$ is the optimal solution to Problem~\ref{eq:optimization_problem_transformed}, then we have for any $\pi \in \mathcal{F}$, $\mathbb{E}_{\pi^*} \left[ R^' (t) \right] \ge \mathbb{E}_{\pi} \left[ R^' (t) \right]$.

\end{document}
